
\section{Протоколы общения дрона и наземной станции}
\subsection{Структура пакета}
Рассмотрим подробнее устройство MAVLink. Ниже приведена структура пакета MAVLink v2. Представление в памяти может отличаться.

uint8\_t magic;              //Метка начала

uint8\_t len;                //Размер данных/длинна полезной нагрузки (сообщения)

uint8\_t incompat\_flags;     //Обратно несовместимые флаги

uint8\_t compat\_flags;       //Обратно совместимые флаги

uint8\_t seq;                //Порядковый номер сообщения для выявления потери сообщения

uint8\_t sysid;              //ID системы-отправителя

uint8\_t compid;             //ID компонента-отправителя

uint8\_t msgid 0:7;          //ID сообщения (первый байт), от него зависит, какие данные будут лежать в полезной нагрузке пакета

uint8\_t msgid 8:15;         //ID сообщения (второй байт)

uint8\_t msgid 16:23;        //ID сообщения (третий байт)

uint8\_t payload[max 255];   //Полезная нагрузка (размер сообщения максимум 255 байт) 

uint16\_t checksum;          //Контрольная сумма

uint8\_t signature[13];      //Сигнатура (опционально)

Структура пакета MAVLink v1 аналогична, но опускает incompat\_flags, compat\_flags и signature, и имеет только один байт для адреса сообщения.


\subsection{Публикация}
Беспроводной формат MAVLink оптимизирован для систем с ограниченными ресурсами и, следовательно, порядок полей не такой, как в спецификации XML. Беспроводной генератор сортирует все поля сообщения по размеру, сначала с самыми большими полями (uint64\_t), а затем с меньшими полями. Сортировка выполняется с использованием стабильного алгоритма сортировки, который гарантирует, что любые поля, которые не нужно переупорядочивать, останутся в том же относительном порядке. Это предотвращает проблемы с выравниванием в системах кодирования / декодирования и позволяет очень эффективно упаковывать / распаковывать.

\subsection{Многоадресные потоки и гарантированная доставка}
MAVLink создан для гибридных сетей, в которых высокоскоростные потоки данных от источников данных (беспилотных летательных аппаратов) поступают в приемники данных (наземные станции), но смешиваются с передачами, требующими гарантированной доставки. Ключевой вывод состоит в том, что для большинства потоков телеметрии не существует известного или единственного получателя: вместо этого, как правило, бортовой компьютер, наземная станция управления и облачная система нуждаются в одном и том же потоке данных.

С другой стороны, настройка бортовой миссии или изменение конфигурации системы с бортовыми параметрами требует точка-точка связи с гарантированной доставкой. MAVLink достигает очень высокой эффективности за счет использования обоих режимов работы.

\subsection{Режим топиков (публикация-подписка)}
%написать про топики и подписки
В режиме топиков протокол не будет выдавать идентификатор целевой системы и компонента для сообщений, чтобы сэкономить пропускную способность канала. Типичными примерами этого режима связи являются все потоки данных автопилота, такие как положение, координаты и т. д.

Основное преимущество этого режима заключается в том, что не создаются дополнительные накладные расходы, и все подписчики могут получать эти данные.
\subsection{Соединение точка-точка}% топология?
Двухточечный режим
В режиме точка-точка MAVLink использует идентификатор цели и целевой компонент. В большинстве случаев, когда используются эти поля, подпротокол также обеспечивает гарантированную доставку (миссии, параметры, команды).

\subsection{Проверки целостности}
MAVLink реализует две проверки целостности: первая проверка целостности пакета во время передачи с использованием контрольной суммы X.25 ( $CRC-16-CCITT$ ). Однако это только гарантирует, что данные не были изменены в ссылке -- это не гарантирует согласованности с определением данных. Вторая проверка целостности связана с описанием данных, чтобы убедиться в содержании идентичной информации у двух сообщений с одинаковым идентификатором. Для этого само определение данных проходит через $CRC-16-CCITT$, а полученное значение используется для заполнения пакета CRC. Большинство эталонных реализаций хранят эту константу в массиве CRC\_EXTRA . \cite{mavlink}


% из хабра

Библиотека MAVLink позволяет кодировать и раскодировать пакеты согласно протоколу, но она не регламентирует, какими аппаратными и программными средствами данные будет отправлены — это могут быть TCP/UDP сообщения, обмен через последовательный порт, - все, что обеспечивает двухсторонний обмен. Библиотека обрабатывает входные данные побайтово, добавляя их в буфер и сама собирает из них пакет. Каждая система или компонент, может одновременно обмениваться данными по разным источникам, тогда для каждого источника назначается специальный идентификатор, называемый channel (канал). MAVLink содержит буфер на каждый канал.

\url{https://habr.com/ru/post/312300/}

Канал связи
Протокол MAVLink может быть использован поверх следующих каналов связи:
последовательное соединение (UART, USB и др.);
UDP (Wi-Fi, Ethernet, 3G, LTE);
TCP (Wi-Fi, Ethernet, 3G, LTE).
Сообщение
MAVLink-сообщение это отдельная "порция" данных, передаваемая между устройствами. Отдельное MAVLink-сообщение содержит информацию о состоянии дрона или команду для дрона.

Примеры MAVLink-сообщений:
ATTITUDE, ATTITUDE\_QUATERNION – ориентация квадрокоптера в пространстве;
LOCAL\_POSITION\_NED – локальная позиция квадрокоптера;
GLOBAL\_POSITION\_INT – глобальная позиция квадрокоптера (широта/долгота/высота);
COMMAND\_LONG – команда для квадрокоптера (взлететь, сесть, переключить режим и т. д.).

\url{https://clover.coex.tech/ru/mavlink.html}

//замечания следующие:
//осознанно переписать
//подвести итог, чем полезно, как улучшить для наших целей(возможно, одна из задач на следующий семестр) итд

\subsection{smthng}
