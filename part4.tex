
\section{Протоколы общения дрона и наземной станции}
\subsection{Структура пакета}
Ключевая особенность разрабатываемого ПАК в том, что вся необходимая для управления информация будет передаваться с помощью протокола MAVLink через средства приема-передачи телеметрии по воздуху, в то время как существующие решения передают информацию через UART.

Рассмотрим подробнее устройство MAVLink. Ниже приведена структура пакета MAVLink v2. Представление в памяти может отличаться.

uint8\_t magic;              //Метка начала

uint8\_t len;                //Размер данных / длина полезной нагрузки (сообщения)

uint8\_t incompat\_flags;     //Обратно несовместимые флаги

uint8\_t compat\_flags;       //Обратно совместимые флаги

uint8\_t seq;                //Порядковый номер сообщения для выявления потери сообщения

uint8\_t sysid;              //ID системы-отправителя

uint8\_t compid;             //ID компонента-отправителя

uint8\_t msgid 0:7;          //ID сообщения (первый байт), от него зависит, какие данные будут лежать в полезной нагрузке пакета

uint8\_t msgid 8:15;         //ID сообщения (второй байт)

uint8\_t msgid 16:23;        //ID сообщения (третий байт)

uint8\_t payload[max 255];   //Полезная нагрузка (размер сообщения максимум 255 байт) 

uint16\_t checksum;          //Контрольная сумма

uint8\_t signature[13];      //Сигнатура (опционально)

Структура пакета MAVLink v1 аналогична, но опускает incompat\_flags, compat\_flags и signature, и имеет только один байт для адреса сообщения \cite{mavlink}.

%https://mavlink.io/en/about/overview.html
\subsection{Публикация}
Беспроводной формат MAVLink оптимизирован для систем с ограниченными ресурсами и, следовательно, порядок полей не такой, как в спецификации XML. Все поля сообщения сортируются по размеру, сначала с самыми большими полями (uint64\_t), а затем с меньшими полями. Сортировка выполняется с использованием стабильного алгоритма сортировки, который гарантирует, что любые поля, которые не нужно переупорядочивать, останутся в том же порядке. Это предотвращает проблемы с выравниванием в системах кодирования / декодирования и позволяет очень эффективно упаковывать / распаковывать данные \cite{mavlink}.

Примеры MAV\-Link-сообщений:
ATTITUDE, ATTITUDE\_QUATERNION – ориентация квадрокоптера в пространстве;
LOCAL\_POSITION\_NED – локальная позиция квадрокоптера;
GLOBAL\_POSITION\_INT – глобальная позиция квадрокоптера (широта/долгота/высота);
COMMAND\_LONG – команда для квадрокоптера (взлететь, сесть, переключить режим и т. д.) \cite{clover}.

\subsection{Многоадресные потоки и гарантированная доставка}
MAVLink создан для систем, в которых высокоскоростные потоки данных от беспилотников поступают в наземные станции, но смешиваются с передачами, требующими гарантированной доставки. Ключевой момент состоит в том, что для большинства потоков телеметрии не существует известного или единственного получателя: вместо этого, как правило, наземная станция управления нуждаются в одном и том же потоке данных.
С другой стороны, настройка бортовой миссии или изменение конфигурации системы с бортовыми параметрами требует точка -- точка связи с гарантированной доставкой. MAVLink достигает очень высокой эффективности за счет использования обоих режимов работы.

\subsection{Соединение точка -- точка}
В режиме точка -- точка при изменении миссии, параметры и передаче команд MAV\-Link использует идентификатор цели и целевой компонент. \cite{mavlink}.

\subsection{Режим топиков (публикация-подписка)}
В режиме топиков протокол не будет выдавать идентификатор целевой системы и компонента для сообщений, чтобы сэкономить пропускную способность канала. Типичными примерами этого режима связи являются все потоки данных автопилота, такие как положение, координаты и т. д.

Основное преимущество этого режима заключается в том, что не создаются дополнительные накладные расходы, и все подписчики могут получать эти данные.

Для наших целей MAVLINK полезен тем, что позволяет получать практически всю информации о внутреннем состоянии полетного контроллера и передавать ему на вход управляющие сигналы. Управляющие сигналы могут быть как в виде указания положения стиков радиоаппаратуры, так и в виде задания угловых скоростей для квадрокоптера. Самым верхнеуровневым типом команды является указание требуемой позиции с предоставлением полетному контроллеру самостоятельного выбора способа достижения этой позиции. Таким образом, мы можем очень точно управлять положением квадрокоптера.
