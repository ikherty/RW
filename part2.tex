% вторая часть

\section{Разработка архитектуры микродрона}
\subsection{Аппаратная часть}

Квадрокоптер состоит из:

--- полетного контроллера,

--- 4 регуляторов оборотов,

--- 4 моторов,

--- рамы,

--- камеры,

--- видеопередатчика,

--- видеоантенны,

--- радиоприемника или телеметрийного модуля

Набор наземной станции и квадрокоптера в основном планируется использовать в помещении. В случае использования БПЛА на улице, при весе свыше 250г требуется регистрация, согласно постановлению такому-то. Основываясь на этом, поставлены следующие условия к компонентам квадрокоптера:

--- размер не должен превышать 140*140*50 \(мм^3\),

--- полетный вес должен быть ниже 250 г,

--- квадрокоптер должен выдерживать столкновения,

--- пропеллеры должны быть защищены,

--- минимальное полетное время 5 мин,

--- энергопотребление?

Подходя к вопросу выбора рамы, стоит учитывать такие факторы как:

--- жесткость рамы,

--- легкий вес,

--- диагональную прочность,

--- стоимость,

--- расстояния между отверстиями, совпадающие с монтажными отверстиями на электронике

Были проведены испытания с рамами из разных материалов. Рассматривались следующие альтернативы: фанера, PLA и PETG пластики, текстолит, углепластик(карбон). Фанера обладает низкой стоимостью, но уступает по жесткости остальным альтернативам.(подумать насчет десернс таблицы). PLA пластик самый безопасный для здоровья человека, им можно печатать детали на 3d принтере, но не устойчив к ударам. PETG обладает большей прочностью по сравнению с ПЛА, но недостаточно жесткий, в связи с чем может вносить осцилляции в гироскоп, ухудшая работу ПИД регулятора полетного контроллера. Текстолит является самым жестким среди вышеперечисленных альтернатив, но обладает самым большим весом. Карбон уступает по стоимости, однако является самым жестким и относительно легким вариантом. Обладает достаточной прочностью, благодаря чему не вносит нежелательные осцилляции. Таким образом, было решено ставить карбоновую раму.
Защита для пропеллеров пластиковая, так как обладает упругостью и низкой стоимостью.

Форм фактор рамы также является немаловажной деталью. Для выполнения задач, где используется направление камеры вниз, вперед и вверх необходимо, чтобы защита пропеллеров, пластины рамы, а также аккумулятор не загромождали обзор. Оптимальным решением является рама с вытянутым корпусом и расположением лучей по типу deadcat (не могу не вставить это в презентацию)- передние лучи разведены на угол, близкий к 180 градусам. Расстояние между отверстиями для монтажа электроники выгоднее выбирать из стандартов - 16*16, 20*20 или 25,5*25,5 мм. Вариант 25,5*25,5мм рассматривать стоит только в том случае, если необходимо использовать плату, где разведены и полетный контроллер и регулятор. В этой работе такая плата неуместна, так как: в случае поломки заменяется полностью, стоимость больше, чем у стека из двух плат, и выбор такой платы с ресурсами, необходимыми для реализации моего проекта, крайне мал. Основываясь на вышеперечисленном была приобретена рама, представленная на рисунке 11.
Электроника квадрокоптера должна быть совместимой по характеристикам и габаритам. Регуляторы оборотов существуют 2х типов: раздельные для каждого мотора и размещенные на одной плате с монтажными отверстиями, как у полетного контроллера. Для легкого БПЛА выгоднее ставить стек из платы с 4 регуляторами и полетного контроллера.
Для управления с наземной станции полетный контроллер должен:

--- обладать минимум 2 UARTами,

--- иметь процессор на базе F405/F745/F765 чипа

Винто-моторная группа должна быть оптимизирована под задачи автономного полета в помещении на небольшой скорости.
Диаметр винта определяет статическую тягу квадрокоптера.
Шаг винта определяет скорость потока воздуха.
Количество лопастей влияет на тяговооруженность.
//Формула 
У бесколлекторных моторов основными параметрами являются размеры статора (4 цифры) и количество оборотов на Вольт(kv). В четырех-значном числе первые два отвечают за диаметр статора, вторые - за высоту статора. Чем больше диаметр статора, тем больше тяговооруженность мотора. Чем больше высота, тем больше скорость
//формула
Оптимальным сочетанием является связка из 1202 6000kv моторов и пропеллеров 2*2.3*3

Видеопередатчик и камера
Видеопередатчик обладает такими характеристиками как:

--- выходная мощность,

--- частота передачи,

--- количество каналов

Для помещений мощность 25mW
Количество каналов должно быть выбрано таким образом, чтобы в случае совместных полетов сигнал не пересекался с чужим.?!
Частота видеосигнала будет использоваться 5.8ГГЦ

Приемник или телеметрийный модуль


\subsection{Программная часть}
В качестве прошивки для квадрокоптера был выбран PX4 - проект с открытым исходным кодом, позволяющий выполнять автономные полеты.
//почему был выбран рх4?
//его возможности?
//estimator (lpe)
