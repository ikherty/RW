Сфера беспилотных и робототехнических систем растет с неимоверной скоростью -- в 2020 году их количество увеличилось вдвое по сравнению с прошлым годом. На рынке труда спрос на специалистов данной области не удовлетворен и только растет. В связи с чем в программу основного общего образования включают курсы программирования и робототехники, закупаются образовательные комплекты на основе квадрокоптеров. \cite{minobr}

Однако комплектов, способных удовлетворить спрос образовательных учреждений, не так много, самыми известными являются Геоскан Пионер, COEX Клевер и NanoPix MiniBot.

Первые два комплекта представляют собой квадрокоптеры диаметром до 300мм с микрокомпьютером на борту, выполняющим все вычислительные операции. Падения таких беспилотников приводят к риску для окружающих, а так же дополнительным тратам, так как ломается дорогостоящее оборудование. Из-за больших размеров возникает потребность в большом помещении, огражденном сеткой. Для Пионера также необходимы дополнительные датчики для его позиционирования, которые необходимо размещать в углах помещения. Эти устройства не входят в комплект и их стоимость близка к стоимости самого квадрокоптера.

NanoPix MiniBot -- миниатюрный комплект, безопасный для окружающих, однако не обладает бортовым компьютером, а на малых вычислительных мощностях не способен выполнять сложные задачи и автономные миссии.

В ходе эксплуатации комплекта Клевер в рамках участия на соревнованиях WorldSkills Russia "Молодые профессионалы" появилась идея как усовершенствовать/модиф сущ комплекты
 
 вывести борт комп в наземную станцию
 все вышеперечисл факторы обусловлен тем что квад должен иметь  борт пк=увел размер и все вытек последств
 и потом первое предло
 
 таким образом суть что будет передаваться сигнал с наземки
 
Идея состоит в том, чтобы вынести квадрокоптера микрокомпьютер, тем самым уменьшив размеры дрона и возможные риски.

когда на борту-ламаеца

Таким образом, получается программно -- аппаратный комплекс из микроквадрокоптера и наземной управляющей станции.
Цель данной научно -- исследовательской работы разработать аппаратную часть такого комплекта, а также описать протоколы взаимодействия наземной станции с квадрокоптером.
