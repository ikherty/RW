Сфера беспилотных и робототехнических систем растет с неимоверной скоростью -- в 2020 году их количество увеличилось вдвое по сравнению с прошлым годом. На рынке труда спрос на специалистов данной области не удовлетворен и только растет. В связи с чем в программу основного общего образования включают курсы программирования и робототехники, закупаются образовательные комплекты на основе квадрокоптеров \cite{minobr}.

Однако комплектов, способных удовлетворить спрос образовательных учреждений, не так много, самыми известными являются Геоскан Пионер, COEX Клевер и NanoPix MiniBot.

Первые два комплекта представляют собой квадрокоптеры диаметром до 300мм с микрокомпьютером на борту, выполняющим все вычислительные операции. Даже штатное функционирование таких аппаратов требует строгого соблюдения техники безопасности ввиду высокой травмоопасности , а внештатные ситуации приводят к серьезным поломкам с высокими затратами на ремонт. Из-за больших размеров возникает потребность в большом помещении, огражденном сеткой. Для автономных полетов внутри помещений набору Пионер требуются дополнительные датчики для его позиционирования, которые необходимо размещать в углах помещения. Эти устройства не входят в комплект и их стоимость близка к стоимости самого квадрокоптера.

NanoPix MiniBot -- миниатюрный комплект, безопасный для окружающих, однако не обладает бортовым компьютером, а на малых вычислительных мощностях не способен выполнять сложные задачи и автономные миссии.

В ходе эксплуатации комплекта COEX Клевер в рамках участия в соревнованиях WorldSkills Russia "Молодые профессионалы" по направлению "Эксплуатация беспилотных авиационных систем" появилась идея, как усовершенствовать существующие комплекты.
Суть идеи заключается в выносе бортового компьютера квадрокоптера в наземную станцию, тем самым существенно уменьшая размер квадрокоптера. Такой квадрокоптер менее опасен, более устойчив к падениям и ударам и его стоимость получается ниже стоимости ближайших конкурентов. Наземная станция помимо компьютера состоит из набора радиомодулей для получения и последующей обработки изображения с квадрокоптера. Результаты обработки отправляются по радио на борт квадрокоптера в виде управляющих сигналов. Таким образом, получается программно -- аппаратный комплекс из микроквадрокоптера и наземной управляющей станции.

Цель данной научно -- исследовательской работы разработать аппаратную часть такого комплекта, а также описать протоколы взаимодействия наземной станции с квадрокоптером.
