
\section{Проверка работоспособности системы}

\url{https://discuss.ardupilot.org/t/indoor-autonomous-flight-with-arducopter-ros-and-aruco-boards-detection/34699}

первоисточник 

\url{https://github.com/ArduPilot/ardupilot\_wiki/blob/master/dev/source/docs/ros-aruco-detection.rst}

\subsection{Настройка ardupilot}
Прошила через qgroundcontrol на pixracer r15 ardupilot, произвела первоначальную настройку-выбрала тип рамы, откалибровала датчики.
Указала следующие параметры:
\begin{MyCode}
AHRS_EKF_TYPE 2
EKF2_ENABLE 1
EKF3_ENABLE 0
EK2_GPS_TYPE 3
EK2_POSNE_M_NSE 0.1
EK2_EXTNAV_DELAY 80
GPS_TYPE 0
COMPASS_USE 0
VISO_TYPE 0
\end{MyCode}
\subsection{Настройка rpi zero w}
Качаем последний образ стретч лайт отсюда:

http://www.pcds.fi/downloads/operatingsystem/debianbased/raspbian/archive/stretch/raspbian.stretch.html
\begin{MyCode}
$ unzip -p 2018-11-13-raspbian-stretch-lite.zip
$ sudo dd if=/home/qw/2018-11-13-raspbian-stretch-lite.img bs=4M of=/dev/sdd conv=fsync
$ sync
\end{MyCode}

\begin{MyCode}
в /etc/wpa_supplicant/wpa_supplicant.conf
укажем
ctrl_interface=DIR=/var/run/wpa_supplicant GROUP=netdev
update_config=1
country=RU

network={
	ssid="ИМЯ_ТОЧКИ_ДОСТУПА"
	psk=123456789
}
\end{MyCode}
Подключаем клавиатуру и монитор к  Raspberry через переходники, логинимся через pi/raspberry.
Поднимаем ssh :
\begin{MyCode}
$ sudo systemctl enable ssh
$ sudo systemctl start ssh
Перезагружаем Raspberry.
Это можно не делать, если создать в /boot файл ssh:
$ sudo touch /boot/ssh
$ reboot
\end{MyCode}
Подключаем малинку, даем минуты 2 на размышления и прогоняем на ПК тулзой nmap, какие устройства к роутеру подключены
\begin{MyCode}
$ sudo nmap -sn 192.168.1.0/24
Находим
Nmap scan report for 192.168.1.148
Host is up (-0.062s latency).
MAC Address: B8:27:EB:D3:B7:09 (Raspberry Pi Foundation)
Nmap scan report for ikherty (192.168.1.28)
Host is up.
Nmap done: 256 IP addresses (4 hosts up) scanned in 4.42 seconds

192.168.1.148-адрес Raspberry.
\end{MyCode}

\begin{MyCode}
	Доставила пакеты:
	gstreamer1.0 (installed with apt)
	gstreamer1.0-plugins-good (иначе ошибка при старте стрима WARNING: erroneous pipeline: no element "rtspsrc")
	gst-rpicamsrc (https://github.com/thaytan/gst-rpicamsrc 27) to use rpicamsrc as source for gst-launch-1.0
	ser2net (installed with apt) a serial to network proxy
	in /etc/ser2net.conf add line:
	2000:raw:0:/dev/ttyAMA0:115200 8DATABITS NONE 1STOPBIT
\end{MyCode}
https://discuss.ardupilot.org/t/indoor-autonomous-flight-with-arducopter-ros-and-aruco-boards-detection/34699/34

https://www.raspberrypi.org/documentation/remote-access/ip-address.md

https://habr.com/ru/post/419947/

Информация о камере:
\begin{MyCode}
$ v4l2-ctl --list-formats-ext -d /dev/video0
\end{MyCode}

Работают, но с задержкой 600мс:
\begin{MyCode}
На rasbberry:
$ raspivid -t 0 -h 720 -w 1080 -fps 25 -hf -b 2000000 -o - | gst-launch-1.0 -v fdsrc ! h264parse !  rtph264pay config-interval=1 pt=96 ! gdppay ! tcpserversink host=192.168.1.148 port=5000 

На ПК: 
$ gst-launch-1.0 -v tcpclientsrc host=192.168.1.148 port=5000  ! gdpdepay !  rtph264depay ! avdec_h264 ! videoconvert ! autovideosink sync=false

https://pi.gbaman.info/?p=150
\end{MyCode}

Уменьшили задержку с помощью передачи udp пакетов. Теперь она 150мс:
Нужна установка gstreamer1.0-tools;
\begin{MyCode}
На ПК запускается:
$ gst-launch-1.0 udpsrc port=5000 ! gdpdepay ! rtph264depay ! avdec_h264 ! videoconvert ! autovideosink sync=false

На Raspberry:
$ gst-launch-1.0 rpicamsrc bitrate=1000000 ! 'video/x-h264,width=640,height=480' ! h264parse ! queue ! rtph264pay config-interval=1 pt=96 ! gdppay ! udpsink host=[IP ПК] port=5000

https://www.raspberrypi.org/forums/viewtopic.php?t=196176
\end{MyCode}

На малине собрали через мейк rpicamsrc,
запустили gst-launch-1.0 -v rpicamsrc bitrate=10000000 rotation=180 exposure-mode=10 awb-mode=0 awb-gain-red=1 awb-gain-blue=2 iso=800 shutter-speed=10000 contrast=50 ! "image/jpeg,width=640,height=480,framerate=30/1" ! udpsink host=192.168.1.148 port=9000


Собрали через каткин драйвер gscam, запустили roscore, осталось изменить лаунч и получить поток.
