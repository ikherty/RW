
\section{p3}


\subsection{Основные сведения}
\subsection{gstreamer}
Утилита gst-launch-1.0 поставляется в пакете gstreamer1.0-tools.

gstreamer -- фреймворк для построения мультимедийных приложений. Практически все в GStreamer является элементом. Все, начиная от обычных источников потоков (filesrc, alsasrc, и т. п.), обработчиками потоков (демультиплексоры, декодеры, фильтры, и т. п.) и заканчивая конечными устройствами вывода (alsasink, fakesink, filesink, и т. п.).
Pad — это некая точка подключения одного элемента к другому, если более просто — это входы и выходы элемента. Обычно они именуются «sink» — вход и «src» — выход.

Жизненный цикл элементы проводят внутри контейнеров. Контейнер управляет рассылкой сообщений от элемента к элементу, управляет статусами элементов. Контейнеры делятся на два вида: Bin и Pipeline.

Pipeline является контейнером верхнего уровня, он управляет синхронизацией элементов, рассылает статусы. Например, если pipeline установить статус PAUSED, этот статус будет автоматически разослан всем элементам которые находятся внутри него. Pipeline является реализацией Bin.
Bin — простой контейнер, который управляет рассылкой сообщений от элемента к элементу которые находятся внутри него. Bin обычно используется для создания группы элементов которые должны совершать какое-либо действие. 

% https://habr.com/ru/post/178813/
Источники данных — это класс плагинов GStreamer, который позволяет читать медиаданные из различных источников, таких как файловая система или аудио-входы звуковой карты. Также, они позволяют получать медиапоток с различных серверов потокового вещания, такие как HTTP (ICECast, ShoutCast), RTSP, RTMP, TCP и UDP. 

Утилита gst-launch-1.0 позволяет запускать GStreamer pipeline без написания кода. Запуск pipeline имеет следующий вид:
gst-launch-1.0 описание-pipeline

а описание pipeline, в свою очередь, делится на описание элементов вида:
element1 property1=value1 property2=value2 ! element2 

Есть элемент типа element1 с свойствами property1 и property2 которые имеют значения value1 и value2 соответственно, и есть элемент типа element2. Символ «!» указывает на то, что выход element1 необходимо соединить с входом element2.
%https://habr.com/ru/post/179167/
В raspberry источником данных является rpicamsrc.

% https://gstreamer.freedesktop.org/
\subsection{rpicamsrc}

Источник видеопотока с Raspberry Pi камеры. rpicamsrc может выводить видео в виде необработанных кадров или закодировать как видео в формате (M)JPEG или H.264 
% https://docs.gstreamer.com/documentation/rpicamsrc/index.html
\subsection{udpsink}
Устройство вывода (sink) — это элемент для вывода сигнала куда-либо:  файл, видеокарта или сетевой интерфейс.. 
% https://habr.com/ru/post/204014/
udpsink - это сетевой приемник, который отправляет UDP-пакеты в сеть.

%https://gstreamer.freedesktop.org/documentation/udp/udpsink.html
\subsection{node}
Нода представляет собой процесс, который выполняет вычисления. Ноды объединяются в граф и взаимодействуют друг с другом с помощью топиков, сервисов RPC и сервера параметров. Ноды предназначены для работы в мелкомасштабном масштабе; система управления роботом обычно состоит из множества нод.

Все ноды имеют имя ресурса графа, которое однозначно идентифицирует их для остальной системы. Ноды также имеют типы, который упрощает процесс обращения к исполняемому файлу узла в файловой системе. Эти типы представляют собой имена ресурсов пакета с именем пакета ноды и именем исполняемого файла ноды. Чтобы определить тип ноды, ROS ищет все исполняемые файлы в пакете с указанным именем и выбирает первый из найденных. 
% http://wiki.ros.org/Nodes
\subsection{nodelet}
Пакет nodelet разработан для обеспечения возможности запуска нескольких нод в одном процессе. Позволяет производить обмен сообщений внутри процесса без затрат на копирование.
http://wiki.ros.org/nodelet
\subsection{topics}
Топиками называют шины, по которым ноды обмениваются сообщениями. Топики имеют семантику анонимной публикации / подписки, которая отделяет производство информации от ее потребления. Как правило, ноды не знают, с кем они общаются. Вместо этого ноды, которые заинтересованы в данных, подписываются на соответствующие топики; ноды, которые генерируют данные, публикуются в соответствующем топике. У топиков может быть несколько издателей и подписчиков.

Топики предназначены для однонаправленного потокового общения. 
Каждый топик строго типизирован в соответствии с типом сообщения ROS, используемым для публикации в ней, и ноды могут получать сообщения только с совпадающим типом. Master не обеспечивает согласованность типа среди издателей, но абоненты не будут устанавливать сообщение транспорта, если топики не совпадают. Кроме того, все клиенты ROS проверяют совпадение MD5, вычисленного из файлов msg. Эта проверка гарантирует, что ноды ROS были скомпилированы из согласованных кодовых баз.

% http://wiki.ros.org/Topics
\subsection{gscam}
gscam -- ROS драйвер, изначально разработанный для трансляции видеопотока на основе gstreamer через стандартный API камеры ROS. Его можно установить из стандартных репозиториев, предоставляемых менеджером apt или собрать вручную с помощью catkin\_make. gscam может быть запущен и как нода, и как нодлет.

gscam может подключаться к специально отформатированному конвейеру. При условии, что этот конвейер обрабатывает видео в формате RGB. gscam ожидает, что переменная окружения GSCAM\_CONFIG будет содержать gstreamerопределение конвейера для его запуска.
gscam получает стрим и публикует 2 топика: camera/image\_raw(необработанное изображение) и camera/camera\_info(содержит калибровку камеры и дополнительные данные о конфигурации камеры).
% http://wiki.ros.org/gscam
\subsection{web\_video\_server}
% http://wiki.ros.org/web_video_server
\subsection{mavros}
см. НИР1.
%http://wiki.ros.org/mavros
\subsection{aruco\_gridboard}

\$ rostopic list
/aruco\_gridboard/camera\_info
/aruco\_gridboard/image\_raw
/mavros/vision\_pose/pose
/vision/pose
/vision/status
\subsection{v4l2sink}
%https://gstreamer.freedesktop.org/documentation/video4linux2/v4l2sink.html
\subsection{как это работает}
The video is streamed from Raspberry Pi with gstreamer and, on the PC, the node gscam get this stream and publish camera/image\_raw and camera/camera\_info topics.
On the PC aruco\_gridboard (slightly modified by me) subscribe to the above topics and publish a camera\_pose message to the mavros/vision\_pose/pose topic.

%https://discuss.ardupilot.org/t/indoor-autonomous-flight-with-arducopter-ros-and-aruco-boards-detection/34699/6
