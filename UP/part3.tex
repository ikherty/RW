
\section{Глава 3. Конфигурация наземной станции}
Наземная станция представляет собой компьютер, подключенный к тому же роутеру, что и дрон. Операционная система на базе ядра linux, установлены пакеты ros, соответствующие версии ОС, в данном случае ros-melodic-full, и gstreamer для приема видеопотока.

\subsection{Настройка mavros}
Для того, чтобы ноды могли обмениваться данными, необходим roscore. roscore - это набор нод и программ, которые являются предпосылками системы на основе ROS \cite{ros}. Запускается с помощью команды \$ roscore в одной из вкладок консоли, однако при использовании roslaunch это действие необязательно -- при выполнении roslaunch первым делом запускается roscore.

roslaunch - это инструмент для простого запуска нескольких нод ROS. Он включает в себя опции для автоматического запуска уже завершенных процессов. roslaunch принимает один или несколько файлов конфигурации XML (с расширением .launch ), определяющих параметры, которые необходимо установить, и ноды для запуска, а также машины, на которых они должны запускаться \cite{ros}.

Для получения телеметрии полетного контроллера в /opt/ros/melodic/sha\-re/mavros/launch/px4.launch файле поменять параметры fcu\_url, указав нужный адрес и порт. Видеопоток планируется получать UDP пакетами, для их обработки необходимо указать адрес и порт в параметре gcs\_url (листинг \ref{lst:9}):
\begin{Program}[H]
	\caption{Измененные параметры в launch файле mavros} \label{lst:9}
\begin{MyCode}
<arg name="fcu_url" default="tcp://192.168.1.148:2000?ids=1,240"/>   
<arg name="gcs_url" default="udp://@127.0.0.1:14555"/>
\end{MyCode}
\end{Program}

\subsection{Подготовка инструментов для получения и обработки видеопотока}
Для получения трансляции и публикации топиков с изображением с камеры используется gscam. Он собирается из репозитория \url{https://github.com/ros-drivers/gscam} командами, представленными в листинге \ref{lst:10}:
\begin{Program}[H]
	\caption{Сборка gscam} \label{lst:10}
\begin{MyCode}
$ git clone https://github.com/ros-drivers/gscam
$ cd gscam
$ cmake -DGSTREAMER_VERSION_1_x=On
$ сmake install
\end{MyCode}
\end{Program}

Распознавание карты aruco маркеров на изображении, получаемом из топиков gscam, и публикацию полученных координат в топик /vision/pose производит aruco\_gridboard. Команды для сборки этого пакета представлены в листинге \ref{lst:11}:
\begin{Program}[H]
	\caption{Сборка aruco\_gridboard} \label{lst:11}
\begin{MyCode}
$ cd ~/catkin_ws/src
$ git clone https://github.com/anbello/aruco_gridboard.git
$ cd ..
$ catkin_make
$ source devel/setup.bash
$ catkin_make --only-pkg-with-deps aruco_gridboard
\end{MyCode}
\end{Program}

Проделанных шагов достаточно, чтобы на наземной станции отобразить положение дрона относительно карты маркеров.


