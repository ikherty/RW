Сфера беспилотных и робототехнических систем растет с неимоверной скоростью -- в 2020 году их количество увеличилось вдвое по сравнению с прошлым годом. На рынке труда спрос на специалистов данной области не удовлетворен и только растет. В связи с чем в программу основного общего образования включают курсы программирования и робототехники, становится популярной концепция STEM-образования, закупаются образовательные комплекты на основе квадрокоптеров \cite{minobr}.

Однако комплектов, способных удовлетворить спрос образовательных учреждений, не так много, самыми известными являются COEX Clover и Геоскан Пионер Мини.

Первый комплект представляет собой квадрокоптер диаметром до 300 мм с микрокомпьютером на борту, выполняющим все вычислительные операции. Даже штатное функционирование такого аппарата требует строгого соблюдения техники безопасности ввиду высокой травмоопасности, а внештатные ситуации приводят к серьезным поломкам с высокими затратами на ремонт. Из-за больших размеров возникает потребность в большом помещении, огражденном сеткой.

Набору Пионер Мини для автономных полетов внутри помещений требуются дополнительные датчики для его позиционирования, которые необходимо размещать в углах помещения. Эти устройства не входят в комплект, и их стоимость превышает стоимость самого квадрокоптера. Помимо этого, проект с 2019 года не развивается.

В ходе эксплуатации комплекта COEX Clover в рамках участия в соревнованиях WorldSkills Russia <<Молодые профессионалы>> по направлению <<Эксплуатация беспилотных авиационных систем>> появилась идея, как усовершенствовать существующие комплекты.

Суть идеи заключается в выносе бортового компьютера квадрокоптера в наземную станцию, тем самым существенно уменьшая размер квадрокоптера. Такой квадрокоптер менее опасен, более устойчив к падениям и ударам и его стоимость получается ниже стоимости ближайших конкурентов. Наземная станция состоит из компьютера и совокупности программных модулей для получения и последующей обработки изображения с квадрокоптера. Результатом обработки являются координаты дрона. За счет wifi-соединения происходит отправка на борт дрона его координат и задание для автономной миссии. Таким образом, получается программно -- аппаратный комплекс из микродрона и наземной управляющей станции.

Цель настоящей работы разработать программно-аппаратный комплекс для создания образовательного робототехнического комплекта на базе БПЛА.
