\section{Разработка архитектуры наземной станции}
Аппаратная часть наземной станции изначально состояла из:
\begin{itemize}
	\item компьютера;
	\item передающего модуля радиоуправления;
	\item видеоприемника;
	\item устройства приема-передачи телеметрии.
\end{itemize}

Планировалось, что наземная станция должна обмениваться телеметрией с квадрокоптером с помощью радиомодулей, получать видеопоток с квадрокоптера через видеоприемник и отправлять управляющий сигнал в виде команд MAVROS с помощью телеметрийных модулей. Чем больше бод -- рейт подключенных модулей и меньше задержка сигнала, тем быстрее осуществляется выполнение команд. Учитывая эти факторы, выбираются устройства телеметрии и видеоприемник.
К компьютеру по UART порту подключался модуль радиоуправления и устройство приема-передачи телеметрии. Через USB порт подключался видеоприемник. Настраивался видеоприемник на диапазон частот, соответствующий частотам видеопередатчика квадрокоптера, и далее следовала настройка программной части.

Однако в ходе выполнения научно-исследовательской работы \cite{nir3} был выявлен более оптимальный способ обмена информации -- по wifi-соединению. Он позволяет как уменьшить задержку, количество компонентов, а значит и стоимость комплекта, так и использовать персональный компьютер в качестве наземной станции, куда необходимо поставить несколько программных решений и настроить взаимодействие между ними.

Таким образом, наземная станция представляет собой компьютер, подключенный к тому же роутеру, что и RPi дрона.
ПО наземной станции представляет собой взаимодействие таких решений как:
\begin{itemize}
	\item операционная система семейства ubuntu версии 16 / 18 на базе ядра linux версии 5.3.0*;
	\item пакетов ros-*;
	\item mavros;
	\item gstreamer;
	\item aruco\_gridboard в связке с gscam.
\end{itemize}

\subsection{Конфигурация наземной станции}

\subsubsection{Настройка MAVROS}
MAVROS (MAVLink + ROS) -- это пакет для ROS, предоставляющий возможность управлять беспилотниками по протоколу MAVLink.  MAVROS подписывается на определенные ROS-топики в ожидании команд, публикует в другие топики телеметрию, и предоставляет сервисы.
Для того, чтобы ноды могли обмениваться данными, необходим roscore \cite{pkg}.

roscore -- это набор нод и программ, которые являются предпосылками системы на основе ROS \cite{ros}. Запускается с помощью команды \$ roscore в одной из вкладок консоли, однако при использовании roslaunch это действие необязательно -- при выполнении roslaunch первым делом запускается roscore.

roslaunch -- это инструмент для простого запуска нескольких нод ROS. Он включает в себя опции для автоматического запуска уже завершенных процессов. roslaunch принимает один или несколько файлов конфигурации XML (с расширением .launch), определяющих параметры, которые необходимо установить, и ноды для запуска, а также машины, на которых они должны запускаться \cite{ros}.

Для получения телеметрии полетного контроллера в /opt/ros/melodic/sha\-re/mavros/launch/px4.launch файле поменять параметры fcu\_url, указав нужный адрес и порт. Видеопоток планируется получать UDP пакетами, для их обработки необходимо указать адрес и порт в параметре gcs\_url (листинг \ref{lst:9}):
\begin{Program}[H]
	\caption{Измененные параметры в launch файле mavros} \label{lst:9}
	\begin{MyCode}
	<arg name="fcu_url" default="tcp://192.168.1.148:2000?ids=1,240"/>   
	<arg name="gcs_url" default="udp://@127.0.0.1:14555"/>
	\end{MyCode}
\end{Program}

\subsubsection{Подготовка инструментов для получения и обработки видеопотока}
Для получения трансляции и публикации топиков с изображением с камеры используется gscam. Он собирается из репозитория \url{https://github.com/ros-drivers/gscam} командами, представленными в листинге \ref{lst:10}:
\begin{Program}[H]
	\caption{Сборка gscam} \label{lst:10}
	\begin{MyCode}
	$ git clone https://github.com/ros-drivers/gscam
	$ cd gscam
	$ cmake -DGSTREAMER_VERSION_1_x=On
	$ сmake install
	\end{MyCode}
\end{Program}

Распознавание карты aruco маркеров на изображении, получаемом из топиков gscam, и публикацию полученных координат в топик /vision/pose производит aruco\_gridboard. Команды для сборки этого пакета представлены в листинге \ref{lst:11}:
\begin{Program}[H]
	\caption{Сборка aruco\_gridboard} \label{lst:11}
	\begin{MyCode}	
	$ cd ~/catkin_ws/src
	$ git clone https://github.com/anbello/aruco_gridboard.git
	$ cd ..
	$ catkin_make
	$ source devel/setup.bash
	$ catkin_make --only-pkg-with-deps aruco_gridboard
	\end{MyCode}
\end{Program}

Проделанных шагов достаточно, чтобы на наземной станции отобразить положение дрона относительно карты маркеров.
