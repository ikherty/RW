
Идея проекта в создании образовательного робототехнического комплекта, состоящего из дрона и наземной станции. 

Суть проекта заключается в модификации и доработке существующих робототехнических решений на основе БПЛА посредством выноса бортового компьютера в наземную станцию, благодаря чему станет возможно уменьшение размеров и стоимости квадрокоптера.
Наземная станция представляет собой совокупность компьютера и радиомодулей, производит преобразование видеосигнала с борта дрона в координаты его положения в пространстве.

Таким образом, квадрокоптер получает локальную систему координат без использования датчиков подобных GPS, а также возможность навигации с использованием компьютерного зрения.
Разработанный ПАК дает возможность создать образовательный робототехнический комплект на основе беспилотника. Вынося бортовой компьютер в наземную станцию, получаем такие преимущества, как низкая стоимость, удешевление ремонта, возможность сделать дрон более компактным, чтобы повысить безопасность.

При выполнении научно-исследовательской работы были разработаны экспериментальные образцы наземной станции и микродрона. В ходе тестирования обнаружилось, что задержка получения аналогового видеосигнала наземной станцией составляла 217 мс. Возникало это из-за задержек, вносимых оборудованием при выполнении аналого-цифрового преобразования на видеоприемнике. В обычном использовании передачи аналогового сигнала с FPV-шлемом задержка составляет 30-50 мс, так как изображение на экране отображается в получаемом формате (PAL или NTSC).

Помимо задержки передачи видеосигнала в накладные расходы будет входить время обработки изображения наземной станцией, а также получения и обработки полетным контроллером управляющей команды на борту квадрокоптера. Учитывая все перечисленное, между фиксацией камерой дрона навигационного маркера и реагированием на него может пройти существенное количество времени. Следствием этого будет перерегулирование, которое с определенной величины задержки перейдет в полную невозможность регулирования положения.
Также одним из явных недостатков аналогового сигнала можно выделить наличие помех на изображении.

Цель данной работы реализовать цифровую видеосистему для микродрона, сравнить ее с аналоговой системой по эксплуатационным характеристикам и провести первые полетные испытания на разработанном микродроне.